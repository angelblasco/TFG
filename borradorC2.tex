\documentclass[a4paper,openright,12pt]{report}
\usepackage[spanish,activeacute]{babel} % espanol
\usepackage[a4paper,margin=2cm]{geometry} % margenes
\usepackage[latin1]{inputenc} % acentos sin codigo
\usepackage{graphicx} % graficos
\usepackage{amsfonts}
\usepackage{float}
\usepackage{amsmath,amssymb,latexsym,stmaryrd}
\numberwithin{equation}{section} % para numerar ecuaciones
\usepackage{fix-cm}
\usepackage{anyfontsize}
\newtheorem{teorema}{Teorema}[section] % para añadir teoremas
\newtheorem{proposicion}{Proposicion}[section] % para añadir proposiciones
\newtheorem{corolario}{Corolario}[section] % para añadir corolarios
\newtheorem{definicion}{Definicion}[section] % para añadir definiciones
\newtheorem{ejemplo}{Ejemplo}[section] % para añadir ejemplos
\newenvironment{proof}{\noindent{\it Demostracion:}}{\hfill$\blacksquare$} % para las demostraciones, con cuadrado negro al final
\setcounter{chapter}{1} % para iniciar un capitulo con una numeracion diferente, el numero debe ser uno menos del numero de capitulo que queremos
\begin{document}
%%%%%% Portada
%\begin{titlepage}

%\begin{center}
%\vspace*{0.6in}
%\begin{figure}[htb]
%\begin{center}
%\includegraphics[width=8cm]{./logo_uned.png}
%\end{center}
%\end{figure}

%\begin{large}
%TRABAJO FIN DE GRADO.\\
%GRADO EN CIENCIAS MATEM'ATICAS.\\
%\end{large}
%\vspace*{0.15in}
%DEPARTAMENTO DE MATEM'ATICAS FUNDAMENTALES. \\
%\vspace*{0.4in}

%\begin{Large}
%\textbf{REPRESENTACI'ON DE GRUPOS FINITOS.} \\
%\end{Large}
%\vspace*{0.3in}

%\rule{80mm}{0.1mm}\\
%\vspace*{0.1in}
%\begin{large}
%Alumno: \\
%Angel Blasco Mu\~noz \\
%\end{large}

%\rule{80mm}{0.1mm}\\
%\vspace*{0.1in}
%\begin{large}
%Tutor:\\
%Dr. Javier Perez Alvarez\\
%\end{large}
%\rule{80mm}{0.1mm}\\

%\end{center}
%\vspace*{1in}
%\begin{flushright}
%Curso 2018 - 2019
%\end{flushright}
%\end{titlepage}
%%%%%%%%%
%Introduccion de pagina en blanco
%\newpage
%$\ $
%\thispagestyle{empty}
%
%%%% Agradecimientos e indice
\addtocontents{toc}{\hspace{-7.5mm} \textbf{Capitulos}}
\addtocontents{toc}{\hfill \textbf{Pagina} \par}
\addtocontents{toc}{\vspace{-2mm} \hspace{-7.5mm} \hrule \par}

%\chapter*{}
\pagenumbering{Roman} % para comenzar la numeracion de paginas en numeros romanos
%\begin{flushright}
%\textit{A Mateo, mi hijo, \\
%por ense\~narme que lo imposible solo tarda un poco mas.\\
%A Bego\~na, mi mujer, \\
%por toda la paciencia que tiene conmigo.\\
%A Angel y Juliana, mis padres, \\
%por todas las oportunidades que me han dado, incluso cuando no las he %merecido.}
%\end{flushright}
%\chapter*{Agradecimientos} 
%Aqu'i los agradecimientos.
%\begin{abstract}
%Este trabajo versa sobre los grupos finitos y sus representaciones. Es importante destacar que en todo el trabajo al referirnos a \textit{grupo} nos referimos a un  \textbf{conjunto finito} en el cual se definir'a una operaci'on que debe cumplir unos condicionantes, al igual que en los grupos algebraicos propiamente dichos. En los grupos finitos se deben cumplir y respetar las mismas propiedades que en los grupos en general.\\
%\\
%Se introducir'a el concepto de representaci'on de un grupo finito, haciendo especial 'enfasis en el concepto de irreducibilidad, asi como en la teor'ia de caracteres para la determinaci'on de todas las representaciones irreducibles de un grupo dado. Se definir'a a su vez la matriz de caracteres y se calcular'an las representaciones irreducibles de algunos grupos finitos. 
%\end{abstract}
\pagenumbering{arabic} % para comenzar la numeracion de paginas en numeros arabigos
%%% El documento se escribe a partir de aqui en cada chapter, section, etc
%\tableofcontents
%%% Capitulo 1
\chapter{Representaciones de grupos.}
Una representaci'on de un grupo finito $G$ nos proporciona una manera de visualizar $G$ como un grupo de matrices. Para ser mas preciso diremos que una representaci'on es un homomorfismo de $G$ en el grupo de matrices invertibles.\\
La estructura de estos homomorfismos y sus propiedades seran objeto de estudio en este capitulo.   
\section{Representaciones de grupos.}
Sea $V$ un espacio vectorial sobre el cuerpo $\mathbb{C}$ de los n'umeros complejos, y sea $GL(V)$ el grupo de isomorfismos de $V$. Un elemento $a \in GL(V)$ es, por definici'on, una aplicaci'on lineal de $V$ en $V$ que admite inversa $a^{-1}$; $a^{-1}$ es tambi'en lineal. Si $V$ admite una base finita $(e_{i})$ de $n$ elementos, toda aplicaci'on lineal $a:V \rightarrow V$ se representa por una matriz cuadrada $(a_{ij})$ de orden $n$. Los coeficientes $a_{ij}$ son n'umeros complejos; se calculan expresando $a(e_{j})$ en la base $(e_{i})$:
\[
a(e_{j})=\sum_{i} a_{ij}e_{i}
\]
Decir que $a$ es un isomorfismo equivale a decir que el determinante de $a$ es no nulo. El grupo $GL(V)$ se identifica as'i como el grupo de matrices cuadradas invertibles de orden $n$. En algunas ocasiones escribiremos $GL(n,V)$.
\begin{definicion}
Sea $G$ un grupo finito. Una representaci'on de $G$ en $V$ es un homomorfismo $\rho$ del grupo $G$ en el grupo $GL(V)$:
\[
\rho : G \rightarrow GL(V)
\]
de modo que:
\[
\rho (st)=\rho (s)  \rho (t)
\]
cualesquiera que sean $s,t \in G$.
\end{definicion}
Supongamos que $V$ es de dimensi'on finita, y sea $n$ su dimensi'on; se dice tambi'en que $n$ es el \textit{grado} de la representaci'on considerada.
\begin{ejemplo}
Sea $G$ el grupo dihedral $D_8=\langle a,b:\, a^{4}=b^{2}=1,\, b^{-1}ab=a^{-1} \rangle$. Definimos las matrices $A$ y $B$ como:
\[
A= \left( \begin{array}{cc}
0 & 1  \\
-1 & 0  \end{array} \right),\,
B= \left( \begin{array}{cc}
1 & 0  \\
0 & -1  \end{array} \right)
\] 
se comprueba que:
\[
A^{4}=B^{2}=I, \, B^{-1}AB=A^{-1}
\]
la funci'on
\[
\rho : G \rightarrow GL(2,V)
\]
definida como $\rho: a^{i}b^{j} \rightarrow A^{i}B^{j}$ para $0\leq i \leq 3$, $0\leq j \leq 1$, es una representaci'on de $D_{8}$ sobre $V$. Es una representaci'on de grado 2.\\
En la siguiente tabla se representan las im'agenes de $\rho$ para cada elemento de $D_{8}$:
\begin{table}[H]
\begin{center}
\begin{tabular}{|c|c|}
\hline
g & $\rho (g)$ \\
\hline \hline
$1$ & $\left( \begin{array}{cc}
1 & 0  \\
0 & 1  \end{array} \right)$ \\ \hline 
$a$ & $\left( \begin{array}{cc}
0 & 1  \\
-1 & 0  \end{array} \right)$ \\ \hline
$a^{2}$ & $\left( \begin{array}{cc}
-1 & 0  \\
0 & -1  \end{array} \right)$ \\ \hline
$a^{3}$ & $\left( \begin{array}{cc}
0 & -1  \\
1 & 0  \end{array} \right)$ \\ \hline
$b$ & $\left( \begin{array}{cc}
1 & 0  \\
0 & -1  \end{array} \right)$ \\ \hline
$ab$ & $\left( \begin{array}{cc}
0 & -1  \\
-1 & 0  \end{array} \right)$ \\ \hline
$a^{2}b$ & $\left( \begin{array}{cc}
-1 & 0  \\
0 & 1  \end{array} \right)$ \\ \hline
$a^{3}b$ & $\left( \begin{array}{cc}
0 & 1  \\
1 & 0  \end{array} \right)$ \\ \hline
\end{tabular}
\caption{Representaci'on del grupo dihedral 8.}
\label{tabla:dihedral8}
\end{center}
\end{table}
\end{ejemplo}
\begin{ejemplo}
Sea $G$ un grupo cualquiera. Definimos $\rho : G \rightarrow GL(n,V)$ como $\rho (g)=I_{n}$ para todo $g \in G$, donde $I_{n}$ es la matriz identidad $n \times n$. Entoces:
\[
\rho (gh) = I_{n}=I_{n}I_{n}=\rho (g)  \rho (h)
\]
para todo $g,h \in G$, por lo tanto, $\rho$ es una representaci'on de $G$. Esto nos indica que todo grupo tiene representaciones de cualquier grado.
\end{ejemplo}
\section{Representaciones equivalentes.}
Sean $\rho$ y $\rho'$ representaciones lineales de un grupo $G$ en espacios vectoriales $V$ y $V'$ respectivamente. Se dice que estas representaciones son equivalentes (o isomorfas) si existe un isomorfismo lineal $\tau : V \rightarrow V'$ que transforma $\rho$ en $\rho'$, es decir, que verifica la identidad:
\[
\tau \cdot \rho (s) = \rho' (s) \cdot \tau
\]
para todo $s \in G$.\\
\\
Si $\rho$ y $\rho'$ se dan en forma matricial por $R$ y $R'$ respectivamente, el isomorfismo se traduce en una matriz invertible $T$ tal que:
\[
T \cdot R = R' \cdot T
\]
o, equivalentemente, tal que:
\[
R' = TRT^{-1}
\] 
Sea $\rho : G \rightarrow GL(V)$ una representaci'on, Y sea $T$ una matriz invertible $n \times n$ de $V$. Para todas las $n \times n$ matrices $A$ y $B$ tenemos:
\[
(T^{-1}AT)(T^{-1}BT)=T^{-1}(AB)T
\]
Usamos esto para crear una reprsentacion $\sigma$ desde $\rho$; definimos
\[
\sigma (g) = T^{-1}\rho (g) T
\]
para todo $g \in G$. Por lo tanto, para todo $g,h \in G$, tenemos:
\[
\sigma (gh) = T^{-1} \rho (gh) T
\]
\[
= T^{-1}\rho (g) \rho (h) T
\]
\[
=T^{-1}\rho (g)T \cdot T^{-1}\rho (h)T 
\]
\[
=\sigma (g) \sigma (h)
\]
por lo que, $\sigma$ es, en efecto, una representaci'on.\\
\\
Con esto podemos ya dar la siguiente definici'on:
\begin{definicion}
Sean $\rho : G \rightarrow GL(m,V)$ y $\sigma : G \rightarrow GL(n,V)$ representaciones de $G$ sobre $V$. Decimos que $\rho$ es equivalente a $\sigma$ si $n=m$ y existe una matriz invertible $n \times n$ $T$ tal que, para todo $g \in G$,
\[
\sigma (g) = T^{-1} \rho (g) T
\]
\end{definicion}
Dadas las representaciones $\rho$, $\sigma$ y $\tau$ de $G$ sobre $V$, se tiene que:
\begin{enumerate}
\item $\rho$ es equivalente a $\rho$. (Prop. Reflexiva).
\item si $\rho$ es equivalente a $\sigma$, entonces $\sigma$ es equivalente a $\rho$. (Prop. Sim'etrica).
\item si $\rho$ es equivalente a $\sigma$ y $\sigma$ es equivalente a $\tau$, entonces $\rho$ es equivalente a $\tau$. (Prop. Transitiva).
\end{enumerate}
Esto nos indica que ser equivalentes es una relaci'on de equivalencia.\\
\\
\begin{proof}
Sean $\rho$, $\sigma$ y $\tau$ representaciones de $G$ sobre $V$, y sea $g \in G$. Tenemos:
\begin{enumerate}
\item Prop. Reflexiva:\\
Sea $I$ la matriz identidad, que ademas es una matriz cuadrada e invertible, siempre podemos poner
\[
\rho (g) = I^{-1} \rho (g) I
\]
por lo que $\rho$ es equivalente a $\rho$.
\item Prop. Sim'etrica:\\
Por ser $\rho$ equivalente a $\sigma$, tenemos que existe una matriz cuadrada invertible $T$ que cumple:
\[
\sigma (g) = T^{-1}\rho (g) T
\]
\[
T \sigma (g) = \rho (g) T
\]
\[
T \sigma (g)T^{-1} = \rho (g) 
\]
lo que concluye que $\sigma$ es equivalente a $\rho$.
\item Prop. Transitiva.\\
Por ser $\rho$ equivalente a $\sigma$, tenemos que existe una matriz cuadrada invertible $T$ que cumple:
\[
\sigma (g) = T^{-1}\rho (g) T
\]
Del mismo modo, por ser $\sigma$ equivalente a $\tau$, tenemos que existe una matriz cuadrada invertible $P$ que cumple:
\[
\tau (g) = P^{-1}\sigma (g) P
\]
Sustituyendo tenemos que:
\[
\tau (g) = P^{-1}T^{-1} \rho (g)T P
\]
\[
\tau (g) = (TP)^{-1} \rho (g)T P
\]
por lo que $\rho$ es equivalente a $\tau$.
\end{enumerate}
\end{proof}
\section{Subrepresentaciones.}
Antes de avanzar en este punto, vamos a tratar, de manera muy breve, algunas nociones relativas a los espacios vectoriales.\\
\\
Sea $V$ un espacio vectorial, $W$ y $W'$ subespacios de $V$. Se dice que $V$ es \textit{suma directa} de $W$ y $W'$ si todo $x \in V$ se puede escribir de manera 'unica en la forma $x=w+w'$, $w \in W$ y $w' \in W'$; equivale a decir que $W \cap W'=0$ y $dim(V)=dim(W)+dim(W')$; se escribe entonces $V=W \oplus W'$, y se dice que $W'$ es \textit{suplementario} de $W$ en $V$. La aplicaci'on $p$ que hace corresponder a cada $x \in V$ su componente $w \in W$ se llama \textit{proyector} de $V$ sobre $W$ (asociado a la descomposici'on $V=W \oplus W'$ ); la imagen de $p$ es $W$, y $p(x)=x$ si $x \in W$; rec'iprocamente, si $p$ es un endomorfismo de $V$ que verifica estas propiedades, inmediatamente se prueba que $V$ es suma directa de $W$ y del n'ucleo $W'$ de $p$. Se establece as'i una correspondencia biyectiva entre los proyectores de $V$ sobre $W$ y los suplementarios de $W$ en $V$.\\
\\
Sea $\rho : G \rightarrow GL(V)$ una representaci'on, y sea $W$ un subespacio de $V$. Si $W$ es estable por la acci'on de $G$, esto es, si $gW \subset W$, $\forall g \in G$, entonces $\rho$ define por restricci'on una representaci'on $\rho' : G \rightarrow GL(W)$.
\section{N'ucleo de una representaci'on.}
Sea una representaci'on $\rho : G \rightarrow GL(V)$. El n'ucleo de una representaci'on consiste en un grupo de elementos $g \in G$ 
para los cuales $\rho (g)$ es la matriz identidad.
\[
Ker \, \rho = \{ g \in G:\rho (g) = I_{n} \}
\]
El n'ucleo de $\rho$ es un subgrupo normal de $G$.\\
\\
Puede ocurrir que el n'ucleo de una representaci'on es el propio grupo $G$.
\begin{definicion}
Una representaci'on $\rho : G \rightarrow GL(1,V)$ definida como:
\[
\rho (g) = 1_{G}
\]
para todo $g \in G$, se denomina \textit{representaci'on trivial} de $G$.
\end{definicion} 
\section{Representaciones irreducibles.}
\begin{definicion}
Una representaci'on lineal $\rho : G \rightarrow GL(V)$ se dice \textit{irreducible} si $V \neq 0$ y ning'un subespacio de $V$ es estable por $G$, excepto, claro est'a, 0 y $V$. 
\end{definicion}
Esto equivale a decir que $V$ no es suma directa de dos subrepresentaciones, salvo la descomposici'on trivial $V=0\oplus V$.\\
\\
Toda representacion de grado 1 es evidentemente irreducible. La suma directa de representaciones irreducibles da cualquier representaci'on.
\begin{teorema}
Toda representaci'on es suma directa de representaciones irreducibles.
\end{teorema} 
\begin{proof}
Sea $V$ una representaci'on lineal de $G$. Se razona por inducci'on sobre $dim(V)$. Si $dim(V)=0$, el teorema es evidente, 0 es suma directa de la familia vacia de representaciones irreducibles. Si $dim(V)\geq 1$ y $V$ es irreducible, tambi'en es cierto el teorema. En el resto de casos podemos descomponer $V$ como suma directa de $V' \oplus V''$, con $dim(V')< dim(V)$ y $dim(V'')< dim(V)$. Por inducci'on, $V'$ y $V''$ son suma directa de representaciones irreducibles y por tanto lo mismo le ocurre a $V$.
\end{proof}\\
\\
Sea $V$ una representaci'on y sea $V=W_{1}\oplus \ldots \oplus W_{k}$ una descomposici'on de $V$ en suma directa de representaciones irreducibles. El numero de las $W_{i}$ isomorfas a una representaci'on irreducible dada no depende de la descomposici'on elegida. 
\section{FG-m'odulos.}
Sea $G$ un grupo, y sea $F=\mathbb{R}$ o $F=\mathbb{C}$. Escribiremos como $V=F^{n}$ el espacio vectorial formado por los vectores fila $(\lambda_{1}, \ldots \lambda_{n})$ con $\lambda_{i} \in F$. Para todo $v \in V$ y $g \in G$, el producto matricial 
\[
v\rho (g)
\]
de el vector fila $v$ con la matriz de dimensi'on $n \times n$ $\rho (g)$, es un vector fila en $V$.\\
\\
Basandonos en el producto matricial $v\rho (g)$ definimos el FG-m'odulo.
\begin{definicion}
Sea $V$ un espacio vectorial sobre $F$ y sea $G$ un grupo. Entonces $V$ es un FG-modulo si esta definida la multiplicaci'on $vg$, para $v \in V$ y $g \in G$, y ademas satisfacen las siguientes condiciones para todo $u, v \in V$, $\lambda \in F$ y $g ,h \in G$:
\begin{enumerate}
\item $vg \in V$
\item $v(gh)=(vg)h$
\item $v1=v$
\item $(\lambda v)g=\lambda (vg)$
\item $(u+v)g=ug+vg$
\end{enumerate}
\end{definicion}
Las condiciones $(1)$, $(4)$ y $(5)$ de la definici'on aseguran que para todo $g \in G$, la funci'on
\[
v \rightarrow vg
\]
es un endomorfismo de $V$.\\
\\
Sea $V$ un FG-m'odulo, y sea $\mathscr{B}$ una base de $V$. Para cada $g \in G$, denotamos como 
\[
[g]_{\mathscr{B}}
\]
a la matriz del endomorfismo $v \rightarrow vg$ de $V$, relativo a la base $\mathscr{B}$.\\
La relaci'on entre los FG-m'odulos y las representaciones de $G$ sobre $F$ se ver'a en el siguiente teorema:
\begin{teorema}
(1) Si $\rho : G \rightarrow GL(F)$ es una representaci'on de $G$ sobre $F$, y $V=F^{n}$, entonces $V$ sera un FG-m'odulo si definimos la multiplicaci'on $vg$ como
\[
vg=v \rho (g)
\]
ademas, existe una base $\mathscr{B}$ de $V$ tal que
\[
\rho (g) = [g]_{\mathscr{B}}
\]
para todo $g \in G$.\\
\\
(2) Sea $V$ un FG-m'odulo y sea $\mathscr{B}$ una base de $V$. Entonces la funci'on 
\[
g \rightarrow [g]_{\mathscr{B}}
\]
es una representaci'on de $G$ sobre $F$. 
\end{teorema}
\begin{proof}
(1) Sabemos que $v \rho (g) \in F^{n}$, ademas por ser $\rho$ un homomorfismo tenemos que $v (\rho (gh))=v(\rho (g) \rho (h))$ y $v (\rho (1))=v$. Del mismo modo, por las propiedades de la multiplicaci'on matricial tenemos que $(\lambda v)\rho(g)=\lambda (v \rho(g))$ y $(u +v)\rho (g)=u \rho(g) + v \rho (g)$ para todo $u, v \in F^{n}$, $\lambda \in F$ y $g, h \in G$.\\
Por lo tanto, $F^{n}$ se convertir'a en un FG-m'odulo si definimos
\[
vg=v \rho (g)
\]
para todo $v \in F^{n}, g \in G$.\\
Ademas, si consideramos la base $\mathscr{B}$ como
\[
(1, 0, 0, ..., 0), (0, 1, 0, ...,0), ..., (0, 0, 0, ... , 1)
\]
de $F^{n}$, entonces $\rho (g) = [g]_{\mathscr{B}}$ para todo $g \in G$.\\
\\
(2) Sea $V$ un FG-m'odulo con base $\mathscr{B}$. De $v(gh)=(vg)h$ para todo $g, h \in G$ y todo $v \in \mathscr{B}$, se sigue que 
\[
[gh]_{\mathscr{B}}=[g]_{\mathscr{B}}[h]_{\mathscr{B}}
\]
En particular, 
\[
[1]_{\mathscr{B}}=[g]_{\mathscr{B}}[g^{-1}]_{\mathscr{B}}
\]
para todo $g \in G$. Ahora $v1=v$ para todo $v \in V$, asi que $[1]_{\mathscr{B}}$ es la matriz identidad.\\
Por lo tanto cada matriz $[g]_{\mathscr{B}}$ es invertible.\\
Hemos probado que la funci'on $g \rightarrow [g]_{\mathscr{B}}$ es un homomorfismo de $G$ a $GL(F)$ y por lo tanto es una representaci'on de $G$ sobre $F$. 
\end{proof}\\
\\
Podemos construir FG-m'odulos sin usar una representaci'on. Para hacer esto transformamos un espacio vectorial $V$ sobre $F$ en un FG-m'odulo especificando la acci'on de los elementos del grupo en una base $v_{1}, \ldots v_{n}$ de $V$ y haciendo que sea lineal la acci'on en el entorno de $V$, es decir, primero definimos $v_{i}g$ para cada $i$ y cada $g \in G$, y entonces definimos
\[
(\lambda_{1}v_{1}+ \ldots +\lambda_{n}v_{n})g
\]
para $\lambda_{i} \in F$, como 
\[
\lambda_{1}(v_{1}g)+ \ldots +\lambda_{n}(v_{n}g)
\]
Como era de esperar existen restricciones a la hora de definir los vectores $v_{i}g$.
\begin{teorema}
Sea $v_{1}, \ldots v_{n}$ una base de un espacio vectorial $V$ sobre $F$. Supongamos que tenemos una multiplicaci'on $vg$ para todo $v \in V$ y $g \in G$, la cual satisface las siguientes condiciones para todo $i$ con $1 \leq i \leq n$, para todo $g, h \in G$ y para todo $\lambda_{1}, \ldots ,\lambda_{n} \in F$:
\begin{enumerate}
\item $v_{i}g \in V$
\item $v_{i}(gh)=(v_{i}g)h$
\item $v_{i}1=v$
\item $(\lambda_{1}v_{1}+ \ldots +\lambda_{n}v_{n})g=\lambda_{1}(v_{1}g)+ \ldots +\lambda_{n}(v_{n}g)$ 
\end{enumerate}
Entonces $V$ es un FG-m'odulo.
\end{teorema}
\begin{proof}
Es trivial ver de (3) y (4) que $v1=v$ para todo $v \in V$. Las condiciones (1) y (4) nos aseguran que, para todo $g \in G$, la funci'on $v \rightarrow vg$ $(v \in V)$ es un endomorfismo de $V$. Esto es:
\[
vg \in V,
\]
\[
(\lambda v)g=\lambda (vg),
\]
\[
(u+v)g=ug+vg,
\]
para todo $u, v \in V$, $\lambda \in F$ y $g \in G$. Por lo tanto
\[
 (\lambda_{1}u_{1}+ \ldots +\lambda_{n}u_{n})h=\lambda_{1}(u_{1}h)+ \ldots +\lambda_{n}(u_{n}h) 
\]
para todo $\lambda_{1}, \ldots ,\lambda_{n} \in F$, todo $u_{1}, \ldots ,u_{n} \in V$, y todo $h \in G$.\\
Ahora sea $v \in V$ y $g, h \in G$. Entonces $v=\lambda_{1}v_{1}+ \ldots +\lambda_{n}v_{n}$ para alg'un $\lambda_{1}, \ldots ,\lambda_{n} \in F$, y
\[
v(gh)=\lambda_{1}(v_{1}(gh))+ \ldots + \lambda_{n}(v_{n}(gh))
\]
\[
=\lambda_{1}((v_{1}g)h)+ \ldots + \lambda_{n}((v_{n}g)h)
\]
\[
=(\lambda_{1}(v_{1}g)+ \ldots + \lambda_{n}(v_{n}g))h
\]
\[
=(vg)h
\]
Con esto hemos comprobado todos los axiomas requeridos para que $V$ sea un FG-m'odulo.
\end{proof}
\begin{definicion}
El FG-m'odulo trivial es un espacio vectorial $V$ sobre $F$ 1-dimensional con
\[
vg=v
\] 
para todo $v \in V$, $g \in G$.
\end{definicion}
\begin{definicion}
Un FG-m'odulo $V$ es fiel si el elemento unitario de $G$ es el 'unico elemento $g$ para el cual
\[
vg=v
\]
para todo $v \in V$.
\end{definicion}
\subsection{FG-m'odulos y representaciones equivalentes.}
Veamos ahora las relaciones entre los FG-m'odulos y las representaciones equivalentes de $G$ sobre $F$. Un FG-m'odulo tiene varias representaciones, todas de la forma
\[
g \rightarrow [g]_\mathscr{B}
\]
para una base $\mathscr{B}$ de $V$. El siguiente resultado muestra que todas estas representaciones son equivalentes entre si.
\begin{teorema}
Sea $V$ un FG-m'odulo con una base $\mathscr{B}$, y sea $\rho$ la representaci'on de $G$ sobre $F$ definida por
\[
\rho: g \rightarrow [g]_\mathscr{B}
\]
(1) Si $\mathscr{B}'$ es una base de $V$, entonces la representaci'on
\[
\phi: g \rightarrow [g]_\mathscr{B'}
\]
de $G$ es equivalente a $\rho$.\\
(2) Si $\sigma$ es una representaci'on de $G$ la cual es equivalente a $\rho$, entonces existe una base $\mathscr{B}''$
de $V$ tal que
\[
\sigma: g \rightarrow [g]_\mathscr{B''}
\]
\end{teorema}
\begin{proof}
(1) Sea $T$ la matriz del cambio de base de $\mathscr{B}$ a $\mathscr{B'}$. Para todo $g \in G$, tenemos
\[
[g]_{\mathscr{B}}=T^{-1}[g]_{\mathscr{B'}}T
\]
Por lo tanto $\phi$ es equivalente a $\rho$.\\
(2) Supongamos que $\rho$ y $\sigma$ son representaciones equivalentes de $G$. Entonces, para la matriz invertible $T$ tenemos
\[
g\rho = T^{-1}(g \sigma )T
\]
para todo $g \in G$. Sea $\mathscr{B''}$ la base de $V$ para la cual la matriz del cambio de base de $\mathscr{B}$ a $\mathscr{B''}$ es $T$. Entonces para todo $g \in G$
\[
[g]_{\mathscr{B}}=T^{-1}[g]_{\mathscr{B''}}T
\]
por lo que $g \sigma = [g]_\mathscr{B''}$.
\end{proof}
\subsection{FG-subm'odulos.}
En lo sucesivo $G$ sera un grupo y $F$ sera $\mathbb{R}$ o $\mathbb{C}$.
\begin{definicion}
Sea $V$ un FG-m'odulo. Un subconjunto $W$ de $V$ se dice que es un FG-subm'odulo de $V$ si $W$ es un subespacio y $wg \in W$ para todo $w \in W$ y $g \in G$. 
\end{definicion}
\subsection{FG-m'odulos irreducibles.}
\begin{definicion}
Un FG-m'odulo $V$ se dice que es irreducible si es diferente a $\{0\}$ y no tiene FG-subm'odulos aparte de $\{0\}$ y $V$.\\
Si $V$ tiene un FG-subm'odulo $W$ donde $W$ es distinto a $\{0\}$ o $V$, entonces $V$ es recucible.
\end{definicion}
Del mismo modo, una representaci'on $\rho : G \rightarrow GL(F)$ es irreducible si el correspondiente FG-m'odulo $F^{n}$ dado por
\[
vg=v(\rho (g))
\]
es irreducible; y $\rho$ es reducible si $F^{n}$ es reducible.\\
\\
Supongamos ahora que $V$ es un FG-m'odulo reducible, por lo tanto hay un FG-subm'odulo $W$ con 0 $<$ dim W $<$ dim V. Tomando una base $\mathscr{B_{1}}$ de $W$ y extendi'endola a una base $\mathscr{B}$ de $V$. Entonces para todo $g \in G$, la matriz $[g]_{\mathscr{B}}$ tiene la forma
\begin{equation}
\left( \begin{array}{c|c}
  X_{g} & 0 \\ 
  \hline
  Y_{g} & Z_{g}
 \end{array} \right)
\end{equation}
para las matrices $X_{g}$, $Y_{g}$ y $Z_{g}$, donde $X_{g}$ es $k \times k$ para $k=dim(W)$.\\
\\
Una representaci'on de grado $n$ es reducible si, y solo si, es equivalente a una representaci'on de la forma (2.6.1), donde $X_{g}$ es $k \times k$ y $0 < k < n$. Notemos que en (2.6.1), las funciones $g \rightarrow X_{g}$ y $g \rightarrow Z_{g}$ son representaciones de $G$.
\subsection{FG-homomorfismos.}
\begin{definicion}
Sean $V$ y $W$ FG-m'odulos. Una funci'on $\vartheta : V \rightarrow W$ se dice que es un FG-homomorfismo si $\vartheta$ es una transformaci'on lineal y 
\[
\vartheta (vg) = \vartheta (v)g
\]
\end{definicion}
En otras palabras, si $\vartheta$ envia $v$ a $w$ entonces envia $vg$ a $wg$.\\
\\
Como $G$ es un grupo finito y $\vartheta : V \rightarrow W$ es un FG-homomorfismo, entonces para todo $v \in V$ y $r=\sum_{g \in G} \lambda_{g}g \in FG$, tenemos
\[
\vartheta (vr) = \vartheta (v)r
\]
\begin{teorema}
Sea $V$ y $W$ FG-m'odulos y sea $\vartheta : V \rightarrow W$ un FG-homomorfismo. Entonces $Ker \, \vartheta$ es un FG-subm'odulo de $V$ y $Im \, \vartheta$ es un FG-subm'odulo de $W$.
\end{teorema}
\begin{proof}
$Ker \, \vartheta$ es un subespacio de $V$ y $Im \, \vartheta$ es un subespacio de $W$ ya que $\vartheta$ es una transformaci'on lineal.\\
Sea $v \in Ker \, \vartheta$ y $g \in G$, entonces
\[
\vartheta (vg) = \vartheta (v)g = 0g = 0
\]
asi que $vg \in Ker \, \vartheta$. Ademas $Ker \, \vartheta$ es un FG-submodulo de $V$.\\
Sea $w \in Im \, \vartheta$, tal que $w = \vartheta (v)$ para alg'un $v \in V$. Para todo $g \in G$,
\[
wg = \vartheta (v)g = \vartheta (vg) \in Im \, \vartheta
\]
por lo que $Im \, \vartheta$ es un FG-subm'odulo de $W$.
\end{proof}
\subsection{Isomorfismos de FG-m'odulos.}
\begin{definicion}
Sean $V$ y $W$ FG-m'odulos. Decimos que $\vartheta : V \rightarrow W$ es un FG-isomorfismo si $\vartheta$ es un FG-homomorfismo y ademas posee inversa. Si $\vartheta : V \rightarrow W$ es un FG-isomorfismo, entonces $V$ y $W$ son FG-m'odulos isomorfos y los representaremos como $V \cong W$.
\end{definicion}
En el siguiente teorema veremos que si $V \cong W$ entonces $W \cong V$.
\begin{teorema}
Si $\vartheta : V \rightarrow W$ es un FG-isomorfismo, entonces la inversa $\vartheta^{-1} : W \rightarrow V$ es tambien un FG-isomorfismo.
\end{teorema}
\begin{proof}
Es evidente que $\vartheta^{-1}$ es una transformaci'on lineal invertible, por lo que, unicamente, debemos demostrar que $\vartheta^{-1}$ es un FG-homomorfismo. Sean $w \in W$ y $g \in G$,
\[
\vartheta(\vartheta^{-1}(w)g)=g(\vartheta(\vartheta^{-1}(w))
\]
como $\vartheta$ es un FG-homomorfismo,
\[
=wg
\]
\[
\vartheta(\vartheta^{-1}(wg))
\]
As'i que $\vartheta^{-1}(w)g=\vartheta^{-1}(wg)$ como se buscaba.
\end{proof}\\
\\
Sea $\vartheta : V \rightarrow W$ un FG-isomorfismo, entonces podemos usar $\vartheta$ y $\vartheta^{-1}$ para cambiar entre los FG-m'odulos isomorfos $V$ y $W$, y probar que $V$ y $W$ comparten la mismas propiedades estructurales; algunos ejemplos pueden ser:
\begin{enumerate}
\item dim $V$ = dim $W$ (cada $v_{1}, \ldots ,v_{n} $ es una base de $V$ si y solo si $\vartheta (v_{1}), \ldots ,\vartheta (v_{n})$ es base de $W$).
\item $V$ es irreducible si y solo si $W$ es irreducible (cada $X$  es un FG-subm'odulo de $V$ si y solo si $\vartheta (X)$ es un FG-subm'odulo de $W$).
\item $V$ contiene un FG-subm'odulo trivial si y solo si $W$ contiene un FG-subm'odulo trivial (cada $X$ es un FG-subm'odulo trivial de $V$ si y solo si $\vartheta (X)$ es un FG-subm'odulo trivial de $W$).
\end{enumerate}
\begin{teorema}
Sea $V$ un FG-m'odulo con una base $\mathscr{B}$, y sea $W$ un FG-m'odulo con una base $\mathscr{B'}$. Entonces $V$ y $W$ son isomorfas si y solo si las representaciones
\[
\rho : g \rightarrow [g]_{\mathscr{B}}
\]
\[
\sigma : g \rightarrow [g]_{\mathscr{B'}}
\]
son equivalentes.
\end{teorema}
\begin{proof}
Para la demostraci'on del anterior teorema primero estableceremos lo siguiente:
\begin{enumerate}
\item Los FG-m'odulos $V$ y $W$ son isomorfos si y solo si existe una base $\mathscr{B}_{1}$ de $V$ y una base $\mathscr{B}_{2}$ de $W$ tal que
\[
[g]_{\mathscr{B}_{1}}=[g]_{\mathscr{B}_{2}}
\]
para todo $g \in G$.
\end{enumerate}
Para ver esto supongamos, primero, que $\vartheta$ es un FG-isomorfismo de $V$ a $W$, y sea $v_{1}, \ldots ,v_{n}$ una base $\mathscr{B}_{1}$ de $V$; entonces $\vartheta (v_{1}), \ldots ,\vartheta (v_{n})$ es una base de $\mathscr{B}_{2}$ de $W$. Sea $g \in G$. Ya que $\vartheta (v_{i}g)=\vartheta (v_{i})g$ para cada $i$, se sigue que $[g]_{\mathscr{B}_{1}}=[g]_{\mathscr{B}_{2}}$.\\
De modo inverso, supongamos que $v_{1}, \ldots ,v_{n}$ una base $\mathscr{B}_{1}$ de $V$ y $w_{1}, \ldots ,w_{n}$ una base $\mathscr{B}_{2}$ de $W$ tal que $[g]_{\mathscr{B}_{1}}=[g]_{\mathscr{B}_{2}}$ para todo $g \in G$. Sea $\vartheta$ la transformaci'on lineal invertible de $V$ a $W$ para la cu'al $\vartheta (v_{i})=w_{i}$. Y que $[g]_{\mathscr{B}_{1}}=[g]_{\mathscr{B}_{2}}$, se deduce que $\vartheta (v_{i}g)=\vartheta (v_{i})g$ y por lo tanto cada $\vartheta$ es un FG-isomorfismo. Esto completa la demostraci'on de (1).\\
Ahora asumimos que $V$ y $W$ son FG-m'odulos isomorfos. Por (1) hay una base $\mathscr{B}_{1}$ de $V$ y una base $\mathscr{B}_{2}$ de $W$ tal que $[g]_{\mathscr{B}_{1}}=[g]_{\mathscr{B}_{2}}$ para todo $g \in G$. Definimos ahora una representaci'on $\phi$ de $G$ como $\phi : g \rightarrow [g]_{\mathscr{B}_{1}}$. Seg'un el teorema 2.6.3(1), $\phi$ es equivalente a $\rho$ y a $\sigma$, por lo que $\rho$ y $\sigma$ son equivalentes.\\
A la inversa, supongamos que $\rho$ y a $\sigma$ son equivalentes, entonces, por el teorema 2.6.3(2) hay una base $\mathscr{B''}$ de $V$ tal que $\sigma (g) = [g]_{\mathscr{B''}}$ para todo $g \in G$; esto es, $[g]_{\mathscr{B'}}=[g]_{\mathscr{B''}}$  para todo $g \in G$. Por lo tanto $V$ y $W$ son FG-m'odulos isomorfos, por (1).
\end{proof}
\subsection{Suma directa de FG-m'odulos.}
Sea $V$ un FG-m'odulo, y supongamos que
\[
V=U \oplus W
\]
donde $U$ y $W$ son FG-subm'odulos de $V$. Sea $u_{1}, \ldots ,u_{m}$ una base $\mathscr{B}_{1}$ de $U$, y $w_{1}, \ldots ,w_{n}$ una base $\mathscr{B}_{2}$ de $W$. Entonces sabemos, por los primeros cursos de 'algebra lineal, que $u_{1}, \ldots ,u_{m},w_{1}, \ldots ,w_{n}$ es una base $\mathscr{B}$ de $V$, y para $g \in G$
\[
\left( \begin{array}{c|c}
  [g]_{\mathscr{B}_{1}} & 0 \\ 
  \hline
  0 & [g]_{\mathscr{B}_{2}}
 \end{array} \right)
\]
Generalizando aun mas, si $V=U_{1} \oplus \ldots \oplus U_{r}$, es suma directa de los FG-subm'odulos $U_{i}$ y $\mathscr{B}_{i}$ es una base de $U_{i}$, entonces podemos unir $\mathscr{B}_{1}, \ldots ,\mathscr{B}_{r}$, para obtener una base $mathscr{B}$ de $V$, y para $g \in G$,
\[
[g]_{\mathscr{B}} =
\left( \begin{array}{ccc}
 [g]_{\mathscr{B}_{1}} & & 0 \\ 
  
   & \ddots &  \\
  0 & & [g]_{\mathscr{B}_{r}}
\end{array} \right) 
\]
\begin{proposicion}
Sea $V$ un FG-m'odulo, y supongamos que
\[
V=U_{1} \oplus \ldots \oplus U_{r}
\]
donde cada $U_{i}$ es un FG-subm'odulo de $V$. Para $v \in V$, tenemos que $v=u_{1}+ \ldots + u_{r}$ para $u_{i} \in U_{i}$, y definimos $\pi_{i}:V \rightarrow V$ como 
\[
\pi_{i}(v)=u_{i}
\]
Entonces cada $\pi_{i}$ es un FG-homomorfismo, y es, ademas, una proyecci'on de $V$.
\end{proposicion}
\begin{proof}
Evidentemente $\pi_{i}$ es una transformaci'on lineal; y es tambi'en un FG-homomorfismo, ya que para $v \in V$ con $v=u_{i}+ \ldots +u_{r}$ siendo $u_{j} \in U_{j}$ para todo $j$, y $g \in G$, tenemos que
\[
\pi_{i}(vg)=\pi_{i}(u_{1}g+ \ldots +u_{r}g)=u_{i}g=\pi_{i}(v)g
\]
por lo tanto 
\[
\pi_{i}^{2}(v)=\pi_{i}(u_{i})=u_{i}=\pi_{i}(v)
\]
asi que $\pi_{i}^{2}=\pi_{i}$. Lo que implica que $\pi_{i}$ es una proyecci'on.
\end{proof}\\
\\
Veamos, por 'ultimo, un resultado concerniente a la suma de FG-m'odulos irreducibles del cual no daremos una demostraci'on.
\begin{proposicion}
Sea $V$ un FG-m'odulo, y supongamos que
\[
V=U_{1}+\ldots +U_{r}
\]
donde cada $U_{i}$ es un FG-subm'odulo irreducible de $V$. Entonces $V$ es una suma directa de algunos FG-subm'odulos $U_{i}$.
\end{proposicion}
\end{document}
