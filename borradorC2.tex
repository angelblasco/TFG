\documentclass[a4paper,openright,12pt]{report}
\usepackage[spanish,activeacute]{babel} % espanol
\usepackage[a4paper,margin=2cm]{geometry} % margenes
\usepackage[latin1]{inputenc} % acentos sin codigo
\usepackage{graphicx} % graficos
\usepackage{amsfonts}
\usepackage{float}
\usepackage{amsmath,amssymb,latexsym,stmaryrd}
\numberwithin{equation}{section} % para numerar ecuaciones
\usepackage{fix-cm}
\usepackage{anyfontsize}
\newtheorem{teorema}{Teorema}[section] % para añadir teoremas
\newtheorem{proposicion}{Proposicion}[section] % para añadir proposiciones
\newtheorem{corolario}{Corolario}[section] % para añadir corolarios
\newtheorem{definicion}{Definicion}[section] % para añadir definiciones
\newtheorem{ejemplo}{Ejemplo}[section] % para añadir ejemplos
\newenvironment{proof}{\noindent{\it Demostracion:}}{\hfill$\blacksquare$} % para las demostraciones, con cuadrado negro al final
\setcounter{chapter}{1} % para iniciar un capitulo con una numeracion diferente, el numero debe ser uno menos del numero de capitulo que queremos
\begin{document}
%%%%%% Portada
%\begin{titlepage}

%\begin{center}
%\vspace*{0.6in}
%\begin{figure}[htb]
%\begin{center}
%\includegraphics[width=8cm]{./logo_uned.png}
%\end{center}
%\end{figure}

%\begin{large}
%TRABAJO FIN DE GRADO.\\
%GRADO EN CIENCIAS MATEM'ATICAS.\\
%\end{large}
%\vspace*{0.15in}
%DEPARTAMENTO DE MATEM'ATICAS FUNDAMENTALES. \\
%\vspace*{0.4in}

%\begin{Large}
%\textbf{REPRESENTACI'ON DE GRUPOS FINITOS.} \\
%\end{Large}
%\vspace*{0.3in}

%\rule{80mm}{0.1mm}\\
%\vspace*{0.1in}
%\begin{large}
%Alumno: \\
%Angel Blasco Mu\~noz \\
%\end{large}

%\rule{80mm}{0.1mm}\\
%\vspace*{0.1in}
%\begin{large}
%Tutor:\\
%Dr. Javier Perez Alvarez\\
%\end{large}
%\rule{80mm}{0.1mm}\\

%\end{center}
%\vspace*{1in}
%\begin{flushright}
%Curso 2018 - 2019
%\end{flushright}
%\end{titlepage}
%%%%%%%%%
%Introduccion de pagina en blanco
%\newpage
%$\ $
%\thispagestyle{empty}
%
%%%% Agradecimientos e indice
\addtocontents{toc}{\hspace{-7.5mm} \textbf{Capitulos}}
\addtocontents{toc}{\hfill \textbf{Pagina} \par}
\addtocontents{toc}{\vspace{-2mm} \hspace{-7.5mm} \hrule \par}

%\chapter*{}
\pagenumbering{Roman} % para comenzar la numeracion de paginas en numeros romanos
%\begin{flushright}
%\textit{A Mateo, mi hijo, \\
%por ense\~narme que lo imposible solo tarda un poco mas.\\
%A Bego\~na, mi mujer, \\
%por toda la paciencia que tiene conmigo.\\
%A Angel y Juliana, mis padres, \\
%por todas las oportunidades que me han dado, incluso cuando no las he %merecido.}
%\end{flushright}
%\chapter*{Agradecimientos} 
%Aqu'i los agradecimientos.
%\begin{abstract}
%Este trabajo versa sobre los grupos finitos y sus representaciones. Es importante destacar que en todo el trabajo al referirnos a \textit{grupo} nos referimos a un  \textbf{conjunto finito} en el cual se definir'a una operaci'on que debe cumplir unos condicionantes, al igual que en los grupos algebraicos propiamente dichos. En los grupos finitos se deben cumplir y respetar las mismas propiedades que en los grupos en general.\\
%\\
%Se introducir'a el concepto de representaci'on de un grupo finito, haciendo especial 'enfasis en el concepto de irreducibilidad, asi como en la teor'ia de caracteres para la determinaci'on de todas las representaciones irreducibles de un grupo dado. Se definir'a a su vez la matriz de caracteres y se calcular'an las representaciones irreducibles de algunos grupos finitos. 
%\end{abstract}
\pagenumbering{arabic} % para comenzar la numeracion de paginas en numeros arabigos
%%% El documento se escribe a partir de aqui en cada chapter, section, etc
%\tableofcontents
%%% Capitulo 1
\chapter{Representaciones de grupos.}
Una representaci'on de un grupo finito $G$ nos proporciona una manera de visualizar $G$ como un grupo de matrices. Para ser mas preciso diremos que una representaci'on es un homomorfismo de $G$ en el grupo de matrices invertibles.\\
La estructura de estos homomorfismos y sus propiedades seran objeto de estudio en este capitulo.   
\section{Representaciones de grupos.}
Sea $V$ un espacio vectorial sobre el cuerpo $\mathbb{C}$ de los n'umeros complejos, y sea $GL(V)$ el grupo de isomorfismos de $V$. Un elemento $a \in GL(V)$ es, por definici'on, una aplicaci'on lineal de $V$ en $V$ que admite inversa $a^{-1}$; $a^{-1}$ es tambi'en lineal. Si $V$ admite una base finita $(e_{i})$ de $n$ elementos, toda aplicaci'on lineal $a:V \rightarrow V$ se representa por una matriz cuadrada $(a_{ij})$ de orden $n$. Los coeficientes $a_{ij}$ son n'umeros complejos; se calculan expresando $a(e_{j})$ en la base $(e_{i})$:
\[
a(e_{j})=\sum_{i} a_{ij}e_{i}
\]
Decir que $a$ es un isomorfismo equivale a decir que el determinante de $a$ es no nulo. El grupo $GL(V)$ se identifica as'i como el grupo de matrices cuadradas invertibles de orden $n$. En algunas ocasiones escribiremos $GL(n,V)$.
\begin{definicion}
Sea $G$ un grupo finito. Una representaci'on de $G$ en $V$ es un homomorfismo $\rho$ del grupo $G$ en el grupo $GL(V)$:
\[
\rho : G \rightarrow GL(V)
\]
de modo que:
\[
\rho (st)=\rho (s)  \rho (t)
\]
cualesquiera que sean $s,t \in G$.
\end{definicion}
Supongamos que $V$ es de dimensi'on finita, y sea $n$ su dimensi'on; se dice tambi'en que $n$ es el \textit{grado} de la representaci'on considerada.
\begin{ejemplo}
Sea $G$ el grupo dihedral $D_8=\langle a,b:\, a^{4}=b^{2}=1,\, b^{-1}ab=a^{-1} \rangle$. Definimos las matrices $A$ y $B$ como:
\[
A= \left( \begin{array}{cc}
0 & 1  \\
-1 & 0  \end{array} \right),\,
B= \left( \begin{array}{cc}
1 & 0  \\
0 & -1  \end{array} \right)
\] 
se comprueba que:
\[
A^{4}=B^{2}=I, \, B^{-1}AB=A^{-1}
\]
la funci'on
\[
\rho : G \rightarrow GL(2,V)
\]
definida como $\rho: a^{i}b^{j} \rightarrow A^{i}B^{j}$ para $0\leq i \leq 3$, $0\leq j \leq 1$, es una representaci'on de $D_{8}$ sobre $V$. Es una representaci'on de grado 2.\\
En la siguiente tabla se representan las im'agenes de $\rho$ para cada elemento de $D_{8}$:
\begin{table}[H]
\begin{center}
\begin{tabular}{|c|c|}
\hline
g & $\rho (g)$ \\
\hline \hline
$1$ & $\left( \begin{array}{cc}
1 & 0  \\
0 & 1  \end{array} \right)$ \\ \hline 
$a$ & $\left( \begin{array}{cc}
0 & 1  \\
-1 & 0  \end{array} \right)$ \\ \hline
$a^{2}$ & $\left( \begin{array}{cc}
-1 & 0  \\
0 & -1  \end{array} \right)$ \\ \hline
$a^{3}$ & $\left( \begin{array}{cc}
0 & -1  \\
1 & 0  \end{array} \right)$ \\ \hline
$b$ & $\left( \begin{array}{cc}
1 & 0  \\
0 & -1  \end{array} \right)$ \\ \hline
$ab$ & $\left( \begin{array}{cc}
0 & -1  \\
-1 & 0  \end{array} \right)$ \\ \hline
$a^{2}b$ & $\left( \begin{array}{cc}
-1 & 0  \\
0 & 1  \end{array} \right)$ \\ \hline
$a^{3}b$ & $\left( \begin{array}{cc}
0 & 1  \\
1 & 0  \end{array} \right)$ \\ \hline
\end{tabular}
\caption{Representaci'on del grupo dihedral 8.}
\label{tabla:dihedral8}
\end{center}
\end{table}
\end{ejemplo}
\begin{ejemplo}
Sea $G$ un grupo cualquiera. Definimos $\rho : G \rightarrow GL(n,V)$ como $\rho (g)=I_{n}$ para todo $g \in G$, donde $I_{n}$ es la matriz identidad $n \times n$. Entoces:
\[
\rho (gh) = I_{n}=I_{n}I_{n}=\rho (g)  \rho (h)
\]
para todo $g,h \in G$, por lo tanto, $\rho$ es una representaci'on de $G$. Esto nos indica que todo grupo tiene representaciones de cualquier grado.
\end{ejemplo}
\section{Representaciones equivalentes.}
Sean $\rho$ y $\rho'$ representaciones lineales de un grupo $G$ en espacios vectoriales $V$ y $V'$ respectivamente. Se dice que estas representaciones son equivalentes (o isomorfas) si existe un isomorfismo lineal $\tau : V \rightarrow V'$ que transforma $\rho$ en $\rho'$, es decir, que verifica la identidad:
\[
\tau \cdot \rho (s) = \rho' (s) \cdot \tau
\]
para todo $s \in G$.\\
\\
Si $\rho$ y $\rho'$ se dan en forma matricial por $R$ y $R'$ respectivamente, el isomorfismo se traduce en una matriz invertible $T$ tal que:
\[
T \cdot R = R' \cdot T
\]
o, equivalentemente, tal que:
\[
R' = TRT^{-1}
\] 
Sea $\rho : G \rightarrow GL(V)$ una representaci'on, Y sea $T$ una matriz invertible $n \times n$ de $V$. Para todas las $n \times n$ matrices $A$ y $B$ tenemos:
\[
(T^{-1}AT)(T^{-1}BT)=T^{-1}(AB)T
\]
Usamos esto para crear una reprsentacion $\sigma$ desde $\rho$; definimos
\[
\sigma (g) = T^{-1}\rho (g) T
\]
para todo $g \in G$. Por lo tanto, para todo $g,h \in G$, tenemos:
\[
\sigma (gh) = T^{-1} \rho (gh) T
\]
\[
= T^{-1}\rho (g) \rho (h) T
\]
\[
=T^{-1}\rho (g)T \cdot T^{-1}\rho (h)T 
\]
\[
=\sigma (g) \sigma (h)
\]
por lo que, $\sigma$ es, en efecto, una representaci'on.\\
\\
Con esto podemos ya dar la siguiente definici'on:
\begin{definicion}
Sean $\rho : G \rightarrow GL(m,V)$ y $\sigma : G \rightarrow GL(n,V)$ representaciones de $G$ sobre $V$. Decimos que $\rho$ es equivalente a $\sigma$ si $n=m$ y existe una matriz invertible $n \times n$ $T$ tal que, para todo $g \in G$,
\[
\sigma (g) = T^{-1} \rho (g) T
\]
\end{definicion}
Dadas las representaciones $\rho$, $\sigma$ y $\tau$ de $G$ sobre $V$, se tiene que:
\begin{enumerate}
\item $\rho$ es equivalente a $\rho$. (Prop. Reflexiva).
\item si $\rho$ es equivalente a $\sigma$, entonces $\sigma$ es equivalente a $\rho$. (Prop. Sim'etrica).
\item si $\rho$ es equivalente a $\sigma$ y $\sigma$ es equivalente a $\tau$, entonces $\rho$ es equivalente a $\tau$. (Prop. Transitiva).
\end{enumerate}
Esto nos indica que ser equivalentes es una relaci'on de equivalencia.\\
\\
\begin{proof}
Sean $\rho$, $\sigma$ y $\tau$ representaciones de $G$ sobre $V$, y sea $g \in G$. Tenemos:
\begin{enumerate}
\item Prop. Reflexiva:\\
Sea $I$ la matriz identidad, que ademas es una matriz cuadrada e invertible, siempre podemos poner
\[
\rho (g) = I^{-1} \rho (g) I
\]
por lo que $\rho$ es equivalente a $\rho$.
\item Prop. Sim'etrica:\\
Por ser $\rho$ equivalente a $\sigma$, tenemos que existe una matriz cuadrada invertible $T$ que cumple:
\[
\sigma (g) = T^{-1}\rho (g) T
\]
\[
T \sigma (g) = \rho (g) T
\]
\[
T \sigma (g)T^{-1} = \rho (g) 
\]
lo que concluye que $\sigma$ es equivalente a $\rho$.
\item Prop. Transitiva.\\
Por ser $\rho$ equivalente a $\sigma$, tenemos que existe una matriz cuadrada invertible $T$ que cumple:
\[
\sigma (g) = T^{-1}\rho (g) T
\]
Del mismo modo, por ser $\sigma$ equivalente a $\tau$, tenemos que existe una matriz cuadrada invertible $P$ que cumple:
\[
\tau (g) = P^{-1}\sigma (g) P
\]
Sustituyendo tenemos que:
\[
\tau (g) = P^{-1}T^{-1} \rho (g)T P
\]
\[
\tau (g) = (TP)^{-1} \rho (g)T P
\]
por lo que $\rho$ es equivalente a $\tau$.
\end{enumerate}
\end{proof}
\section{Subrepresentaciones.}
Antes de avanzar en este punto, vamos a tratar, de manera muy breve, algunas nociones relativas a los espacios vectoriales.\\
\\
Sea $V$ un espacio vectorial, $W$ y $W'$ subespacios de $V$. Se dice que $V$ es \textit{suma directa} de $W$ y $W'$ si todo $x \in V$ se puede escribir de manera 'unica en la forma $x=w+w'$, $w \in W$ y $w' \in W'$; equivale a decir que $W \cap W'=0$ y $dim(V)=dim(W)+dim(W')$; se escribe entonces $V=W \oplus W'$, y se dice que $W'$ es \textit{suplementario} de $W$ en $V$. La aplicaci'on $p$ que hace corresponder a cada $x \in V$ su componente $w \in W$ se llama \textit{proyector} de $V$ sobre $W$ (asociado a la descomposici'on $V=W \oplus W'$ ); la imagen de $p$ es $W$, y $p(x)=x$ si $x \in W$; rec'iprocamente, si $p$ es un endomorfismo de $V$ que verifica estas propiedades, inmediatamente se prueba que $V$ es suma directa de $W$ y del n'ucleo $W'$ de $p$. Se establece as'i una correspondencia biyectiva entre los proyectores de $V$ sobre $W$ y los suplementarios de $W$ en $V$.\\
\\
Sea $\rho : G \rightarrow GL(V)$ una representaci'on, y sea $W$ un subespacio de $V$. Si $W$ es estable por la acci'on de $G$, esto es, si $gW \subset W$, $\forall g \in G$, entonces $\rho$ define por restricci'on una representaci'on $\rho' : G \rightarrow GL(W)$.
\section{N'ucleo de una representaci'on.}
Sea una representaci'on $\rho : G \rightarrow GL(V)$. El n'ucleo de una representaci'on consiste en un grupo de elementos $g \in G$ 
para los cuales $\rho (g)$ es la matriz identidad.
\[
Ker \, \rho = \{ g \in G:\rho (g) = I_{n} \}
\]
El n'ucleo de $\rho$ es un subgrupo normal de $G$.\\
\\
Puede ocurrir que el n'ucleo de una representaci'on es el propio grupo $G$.
\begin{definicion}
Una representaci'on $\rho : G \rightarrow GL(1,V)$ definida como:
\[
\rho (g) = 1_{G}
\]
para todo $g \in G$, se denomina \textit{representaci'on trivial} de $G$.
\end{definicion} 
\section{Representaciones irreducibles.}
\begin{definicion}
Una representaci'on lineal $\rho : G \rightarrow GL(V)$ se dice \textit{irreducible} si $V \neq 0$ y ning'un subespacio de $V$ es estable por $G$, excepto, claro est'a, 0 y $V$. 
\end{definicion}
Esto equivale a decir que $V$ no es suma directa de dos subrepresentaciones, salvo la descomposici'on trivial $V=0\oplus V$.\\
\\
Toda representacion de grado 1 es evidentemente irreducible. La suma directa de representaciones irreducibles da cualquier representaci'on.
\begin{teorema}
Toda representaci'on es suma directa de representaciones irreducibles.
\end{teorema} 
\begin{proof}
Sea $V$ una representaci'on lineal de $G$. Se razona por inducci'on sobre $dim(V)$. Si $dim(V)=0$, el teorema es evidente, 0 es suma directa de la familia vacia de representaciones irreducibles. Si $dim(V)\geq 1$ y $V$ es irreducible, tambi'en es cierto el teorema. En el resto de casos podemos descomponer $V$ como suma directa de $V' \oplus V''$, con $dim(V')< dim(V)$ y $dim(V'')< dim(V)$. Por inducci'on, $V'$ y $V''$ son suma directa de representaciones irreducibles y por tanto lo mismo le ocurre a $V$.
\end{proof}\\
\\
Sea $V$ una representaci'on y sea $V=W_{1}\oplus \ldots \oplus W_{k}$ una descomposici'on de $V$ en suma directa de representaciones irreducibles. El numero de las $W_{i}$ isomorfas a una representaci'on irreducible dada no depende de la descomposici'on elegida.
\end{document}
