\documentclass[a4paper,openright,12pt]{report}
\usepackage[spanish,activeacute]{babel} % espanol
\usepackage[a4paper,margin=2cm]{geometry} % margenes
\usepackage[latin1]{inputenc} % acentos sin codigo
\usepackage{graphicx} % graficos
\usepackage{multirow} % para las tablas
\usepackage{amsfonts}
\usepackage{mathrsfs} % para letras de rubrica
\usepackage{float}
\usepackage{longtable} % Permite uso del entrono longtable
\usepackage{booktabs} % Permite el uso de \toprule, \midrule y \bottonrule para realzar el acabado de las tablas
\usepackage[spanish]{babel} % Captions y sections en español
\usepackage[T1]{fontenc}
\usepackage{amsmath,amssymb,latexsym,stmaryrd}
\numberwithin{equation}{section} % para numerar ecuaciones, utilizar \begin{equation} ... \end{equation}
\usepackage{fix-cm}
\usepackage{anyfontsize}
\newtheorem{teorema}{Teorema}[section] % para añadir teoremas
\newtheorem{proposicion}{Proposicion}[section] % para añadir proposiciones
\newtheorem{corolario}{Corolario}[section] % para añadir corolarios
\newtheorem{definicion}{Definicion}[section] % para añadir definiciones
\newtheorem{ejemplo}{Ejemplo}[section] % para añadir ejemplos
\newenvironment{proof}{\noindent{\it Demostracion:}}{\hfill$\blacksquare$} % para las demostraciones, con cuadrado negro al final
\begin{document}
%%%%%% Portada
%\begin{titlepage}

%\begin{center}
%\vspace*{0.6in}
%\begin{figure}[htb]
%\begin{center}
%\includegraphics[width=8cm]{./logo_uned.png}
%\end{center}
%\end{figure}

%\begin{large}
%TRABAJO FIN DE GRADO.\\
%GRADO EN CIENCIAS MATEM'ATICAS.\\
%\end{large}
%\vspace*{0.15in}
%DEPARTAMENTO DE MATEM'ATICAS FUNDAMENTALES. \\
%\vspace*{0.4in}

%\begin{Large}
%\textbf{REPRESENTACI'ON DE GRUPOS FINITOS.} \\
%\end{Large}
%\vspace*{0.3in}

%\rule{80mm}{0.1mm}\\
%\vspace*{0.1in}
%\begin{large}
%Alumno: \\
%Angel Blasco Mu\~noz \\
%\end{large}

%\rule{80mm}{0.1mm}\\
%\vspace*{0.1in}
%\begin{large}
%Tutor:\\
%Dr. Javier Perez Alvarez\\
%\end{large}
%\rule{80mm}{0.1mm}\\

%\end{center}
%\vspace*{1in}
%\begin{flushright}
%Curso 2018 - 2019
%\end{flushright}
%\end{titlepage}
%%%%%%%%%
%Introduccion de pagina en blanco
%\newpage
%$\ $
%\thispagestyle{empty}
%
%%%% Agradecimientos e indice
\addtocontents{toc}{\hspace{-7.5mm} \textbf{Capitulos}}
\addtocontents{toc}{\hfill \textbf{Pagina} \par}
\addtocontents{toc}{\vspace{-2mm} \hspace{-7.5mm} \hrule \par}

%\chapter*{}
\pagenumbering{Roman} % para comenzar la numeracion de paginas en numeros romanos
%\begin{flushright}
%\textit{A Mateo, mi hijo, \\
%por ense\~narme que lo imposible solo tarda un poco mas.\\
%A Bego\~na, mi mujer, \\
%por toda la paciencia que tiene conmigo.\\
%A Angel y Juliana, mis padres, \\
%por todas las oportunidades que me han dado, incluso cuando no las he %merecido.}
%\end{flushright}
%\chapter*{Agradecimientos} 
%Aqu'i los agradecimientos.
%\begin{abstract}
%Este trabajo versa sobre los grupos finitos y sus representaciones. Es importante destacar que en todo el trabajo al referirnos a \textit{grupo} nos referimos a un  \textbf{conjunto finito} en el cual se definir'a una operaci'on que debe cumplir unos condicionantes, al igual que en los grupos algebraicos propiamente dichos. En los grupos finitos se deben cumplir y respetar las mismas propiedades que en los grupos en general.\\
%\\
%Se introducir'a el concepto de representaci'on de un grupo finito, haciendo especial 'enfasis en el concepto de irreducibilidad, asi como en la teor'ia de caracteres para la determinaci'on de todas las representaciones irreducibles de un grupo dado. Se definir'a a su vez la matriz de caracteres y se calcular'an las representaciones irreducibles de algunos grupos finitos. 
%\end{abstract}
\pagenumbering{arabic} % para comenzar la numeracion de paginas en numeros arabigos
%%% El documento se escribe a partir de aqui en cada chapter, section, etc
%\tableofcontents
%%% Capitulo 1
%%%\chapter{Capitulo 3.}
\chapter{Caracteres.}
\section{El lema de Schur.}
\begin{teorema}[Lema de Schur.]
Sea $f : V \rightarrow V'$ una aplicaci'on lineal entre G-m'odulos irreducibles. Entonces, o bien $f = 0$, o bien $f$ es un isomorfismo; ademas, en este caso, $f$ es una homotecia.
\end{teorema}
\begin{proof}
Como Ker(f) es un $G$-subm'odulo estable de $V$, o bien Ker(f)=$V$, lo que significa que $f=0$, o bien Ker(f)=0. En este caso la condici'on $f(gv)=gf(v)$, para $v \in V$, implica que Im(f) es tambi'en un $G$-subm'odulo invariante de $V'$, con lo que Im(f)=$V'$. Sobre los numeros complejos $f$ ha de tener alg'un valor propio $\lambda$, ello supone que $f - \lambda Id$ tiene n'ucleo no nulo, con lo que por lo anterior  $f - \lambda Id$ es el morfismo nulo y por lo tanto $f = \lambda Id$.
\end{proof}
\begin{corolario}
Toda representaci'on irreducible $V$ de un grupo abeliano es de grado 1.
\end{corolario}
\begin{proof}
Sea $g$ un elemento de $G$ y consideremos la aplicaci'on lineal que induce $g : V \rightarrow V$. Puesto que para todo $h \in G$ se cumple que $gh=hg$, se tiene que $g$ es un morfismo de $G$-m'odulos, con ello $g$ es una homotecia. Asi la representaci'on de $G$ en $V$ convierte a $G$ en un grupo de homotecias. Puesto que una homotecia deja invariante cualquier espacio, si $V$ es irreducible ha de ser de dimensi'on 1.
\end{proof}
\section{Car'acter de una representaci'on.}
Para comprender lo que es el car'acter de una representaci'on necesitamos conocer primero el concepto de traza. Este concepto se estudi'o en los primeros cursos de Algebra Lineal, pero debido a su importancia en esta secci'on procedemos, de nuevo, a dar su definici'on:
\begin{definicion}
Sea $V$ un espacio vectorial de dimension $n$ y $a$ un endomorfismo, cuya matriz, en una base $(e_{i})$ de $V$, es $(a_{ij})$. La traza de $a$ es el escalar 
\[
Tr(a)=\sum_{i} a_{ii}
\]
la traza de $a$ no depende de la base $(e_{i})$ elegida.
\end{definicion}
Sea $\rho : G \rightarrow GL(V)$ una representaci'on lineal de un grupo finito $G$ en el espacio vectorial $V$. Dado $s \in G$, pongamos 
\[
\chi_{\rho}(s)=Tr(\rho(s))
\]
Se obtiene as'i una aplicaci'on $\chi_{\rho}$ definida en $G$, a valores complejos
\[
\chi_{\rho} : G \rightarrow \mathbb{C} 
\]    
llamada \emph{car'acter} de la representaci'on $\rho$, la importancia de esta aplicaci'on proviene de que caracteriza la representaci'on considerada.
\begin{proposicion}
Si $\chi$ es el car'acter de una representaci'on $\rho$ de grado $n$, entonces,
\begin{enumerate}
\item $\chi (1)=n$
\item $\chi (s^{-1})=\chi (s)^{*}$ para todo $s \in G$
\item $\chi (tst^{-1})=\chi(s)$ cualquiera que sean $s \, ,t \in G $
\end{enumerate} 
(Si $x$ es un n'umero complejo, denotamos a su conjugado por $x^{*}$.)
\end{proposicion} 
\begin{proof}
Como $\rho (1) = 1$ y $Tr(1)=n$ por ser $V$ de dimensi'on $n$, tendremos $\chi (1)=n$.\\
\\
$\rho (s)$ es de orden finito; sus valores propios $\lambda_{1}, \ldots .\lambda_{n}$ tambien ser'an de orden finito y por lo tanto de m'odulo 1. Entonces:
\[
\chi (s^{*})=Tr(\rho (s^{*}))=\sum \lambda_{i}^{*}= \sum \lambda_{i}^{-1}=Tr(\rho (s^{-1}))=\chi (s^{-1})
\]
\end{proof}\\
\\
Una aplicaci'on $f$ definida en $G$ que verifica la identidad 3), o, equivalentemente, la identidad $f(uv)=f(vu)$, se llama una \emph{funci'on central}. 
\begin{proposicion}
Sean $\rho_{1}:G \rightarrow GL(V_{1})$ y  $\rho_{2}:G \rightarrow GL(V_{2})$, dos representaciones lineales de $G$, y sean $\chi_{1}$ y $\chi_{2}$ sus caracteres. Entonces
\begin{enumerate}
\item El car'acter $\chi$ de la representaci'on suma directa $V_{1} \oplus V_{2}$ es igual a $\chi_{1} + \chi_{2}$.
\item El car'acter $\psi$ de la representaci'on producto tensorial $V_{1} \otimes V_{2}$ es igual a $\chi_{1} \cdot \chi_{2}$.
\end{enumerate}
\end{proposicion}
\begin{proof}
Expresamos $\rho_{1}$ y $\rho_{2}$ en forma matricial: $R_{s}^{1}$ y $R_{s}^{2}$. La forma matricial de la representaci'on $V_{1} \oplus V_{2}$ es 
\[
R_{s}= \left(
\begin{array}{cc}
R_{s}^{1} & 0 \\
0 & R_{s}^{2}
\end{array}
\right)
\]
de donde $Tr(R_{s})=Tr(R_{s}^{1})+Tr(R_{s}^{2})$, es decir, $\chi(s)=\chi_{1}(s)+\chi_{2}(s)$. Sabemos que:
\[
\chi_{1}(s)=\sum_{i_{1}}r_{i1i1}(s)
\]
\[
\chi_{2}(s)=\sum_{i_{2}}r_{i2i2}(s)
\]
\[
\psi(s)=\sum_{i1,i2}r_{i1i1}(s)\cdot r_{i2i2}(s)=\chi_{1}(s) \cdot \chi_{2}(s)
\]
\end{proof}\\
\\
Veamos esto con un ejemplo.
\begin{ejemplo}
Sea $G=D_{8} = \langle a, b; a^{4}=b^{2}=1, b^{-1}ab=a^{-1}\rangle$, y sea $\rho : G \rightarrow GL(2, \mathbb{C})$ la representaci'on definida como
\[
\rho (a)= \left(
\begin{array}{cc}
0 & 1 \\
-1 & 0
\end{array}
\right)
\]
\[
\rho (b)= \left(
\begin{array}{cc}
1 & 0 \\
0 & -1
\end{array}
\right)
\]
Sea $\chi$ el car'acter de esta representaci'on. En la siguiente tabla podemos ver los valores que toma $\rho (g)$ y $\chi (g)$ en funci'on de $g$.
\begin{longtable}[c]{c c c }
\caption{Valores de los caracteres.} \\ \toprule
\textbf{$g$} & \textbf{$\rho (g)$} &\textbf{$\chi (g)$} \\ \toprule % Es la fila que se repetirá en cada página                  
\endfirsthead
 \multicolumn{3}{c}{\textit{\textsl{(Viene de la página anterior)}}} \\
\caption{Mi primera tabla larga} \\ \toprule
\textbf{columna 1} & \textbf{columna 2} &\textbf{ columna 3} \\ \toprule % Repetirá la fila en todas las páginas                      
\endhead
 %\multicolumn{3}{c}{\textsl{\textit{(Contin'ua en la p'agina siguiente)}}}\\ %\midrule                                                   
\endfoot
 %\multicolumn{3}{c}{\textit{\textsl{(Fin de mi primera tabla larga)}}} \\% \midrule                                                    
%\endlastfoot
1 & $\left(
\begin{array}{cc}
1 & 0 \\
0 & 1
\end{array}
\right)$ & 2 \\ \midrule 
a & $\left(
\begin{array}{cc}
0 & 1 \\
-1 & 0
\end{array}
\right)$ & 0 \\ \midrule
$a^{2}$ & $\left(
\begin{array}{cc}
-1 & 0 \\
0 & -1
\end{array}
\right)$ & -2 \\ \midrule 
$a^{3}$ & $\left(
\begin{array}{cc}
0 & -1 \\
1 & 0
\end{array}
\right)$ & 0 \\ \midrule
b & $\left(
\begin{array}{cc}
1 & 0 \\
0 & -1
\end{array}
\right)$& 0 \\ \midrule 
ab & $\left(
\begin{array}{cc}
0 & -1 \\
-1 & 0
\end{array}
\right)$ & 0 \\ \midrule
$a^{2}b$ & $\left(
\begin{array}{cc}
-1 & 0 \\
0 & 1
\end{array}
\right)$ & 0 \\ \midrule 
$a^{3}b$ & $\left(
\begin{array}{cc}
0 & 1 \\
1 & 0
\end{array}
\right)$ & 0 \\ \bottomrule
\end{longtable}
\end{ejemplo}
\section{Relaciones de ortogonalidad de los caracteres y descomposici'on de la representaci'on regular.}
Introduzcamos la siguiente notaci'on: si $\varphi$ y $\psi$ son funciones del tipo $\varphi , \psi : G \rightarrow \mathbb{C}$, pondremos
\[
\langle \varphi , \psi \rangle = \frac{1}{|G|}\sum_{t \in G} \varphi (t)^{*} \psi (t)
\]
siendo $|G|$ el orden de $G$. Esta expresi'on es un producto escalar.
\begin{teorema}
Si $\chi$ es el car'acter de una representaci'on irreducible, $\langle \chi , \chi \rangle =1$. Si $\chi$ y $\chi'$ son los caracrteres de dos representaciones irreducibles no isomorfas, entonces $\langle \chi , \chi' \rangle=0$, es decir, son ortogonales.
\end{teorema}
\begin{proof}
Sea $\rho: G \rightarrow GL(V)$ la representaci'on irreducible cuyo car'acter es $\chi$, y sea $n$ su grado. Entonces:
\[
\langle \chi , \chi \rangle = \frac{1}{|G|}\sum_{t \in G} \chi (t)^{*} \chi (t)  = \frac{1}{|G|}\sum_{t \in G} \chi (t^{-1}) \chi (t) 
\]
Ahora bien, si ponemos $\rho$ en forma matricial $\rho(t)=[r_{ij}(t)$, $\chi (t) = \sum_{i} r_{ii}(t)$, y por lo tanto
\[
\langle \chi , \chi \rangle = \frac{1}{|G|}\sum_{t, i, j} r_{ii} (t^{-1}) r_{jj} (t) 
\]
lo que implica que
\[
\frac{1}{|G|}\sum_{t \in G} r_{ii} (t^{-1}) r_{jj} (t)
\]
es igual a 0 si $i \neq j$ y a $\frac{1}{n}$ si $i=j$.
\end{proof}
\begin{teorema}
Sea $V$ una representaci'on lineal de $G$, de car'acter $\varphi$, descomponemos $V$ en suma directa de representaciones irreducibles
\[
V = W_{1} \oplus \ldots \oplus W_{k}
\]
Entonces, si $W$ es una representaci'on irreducible de car'acter $\chi$, el numero de las $W_{i}$ isomorfas a $W$ es igual al prducto escalar $\langle \varphi , \chi \rangle$. 
\end{teorema}
Sea $\chi_{i}$ el caracter de $W_{i}$, sabemos que 
\[
\varphi = \chi_{1}+ \ldots + \chi_{k}
\]
y por lo tanto $\langle \varphi , \chi \rangle= \langle \chi_{1}, \chi \rangle +\ldots +   \langle \chi_{k}, \chi \rangle $. Por el teorema anterior $\langle \chi_{i}, \chi \rangle$ es igual a 1 (o a 0) segun que $W_{i}$ sea (o no) isomorfa a $W$.\\
\\
Del anterior teorema podemos afirmar que el n'umero de las $W_{i}$ isomorfas a $W$ no depende de la descomposici'on elegida; ademas, dos representaciones del mismo car'acter son isomorfas, ya que ambas contienen el mismo n'umero de veces toda representaci'on irreducible dada.\\
Estos resultados permiten reducir el estudio de las representaciones al de los caracteres. Si $\chi_{1}, \ldots ,\chi_{h}$ son los caracteres de las representaciones irreducibles de $G$ y $W_{1}, \ldots W_{h}$ son sus correspondientes representaciones, toda representaci'on $V$ es isomorfa a una suma directa
\[
V=m_{1}W_{1} \oplus \ldots \oplus m_{h}W_{h}
\]
para $m_{i}$ entero mayor o igual a 0.\\
El car'acter $\varphi$ de $V$ es igual a $m_{1}\chi_{1}+\ldots + m_{h}\chi_{h}$, y $m_{i}=\langle \varphi , \chi_{i} \rangle$. Las relaciones de ortogonalidad entre los $\chi_{i}$ implican que
\[
\langle \varphi , \varphi \rangle = \sum_{i=1}^{h}m_{i}^{2}
\]
de donde obtenemos el siguiente teorema.
\begin{teorema}
Si $\varphi$ es el car'acter de una representaci'on $V$, $\langle \varphi, \varphi \rangle$ es un entero y $\langle \varphi, \varphi \rangle =1$ si y s'olo si $V$ es irreducible.
\end{teorema}
\begin{proof}
En efecto, $\sum m_{i}^{2}$ vale 1 si y solo si uno de los $m_{i}$ es igual a 1 y los demas iguales a 0, es decir, si y solo si $V$ es isomorfo a una de las $W_{i}$.
\end{proof}\\
\\
Se ha obtenido as'i un \emph{criterio de irreducibilidad} muy c'omodo.\\
\\
Con los siguientes resultados describiremos la descomposici'on de una representaci'on regular:
\begin{proposicion}
El car'acter $\varphi$ de la representaci'on regular viene dado por las siguientes f'ormulas:
\[
\varphi (1) = |G|
\]
donde $|G|$ es el ord'en de $G$. Y
\[
\varphi (s)=0
\]
si $s \neq 1$.
\end{proposicion}
\begin{corolario}
Cada representaci'on irreducible $W_{i}$ est'a contenida en la representaci'on regular un n'umero de veces igual a su grado $n_{i}$.
\end{corolario}
\begin{proof}
Por el teorema 3.3.2, este n'umero es igual a $\langle \varphi , \chi_{i} \rangle$. Ahora bien
\[
\langle \varphi , \chi_{i} \rangle = \frac{1}{|G|}\sum_{s \in G} \varphi (s)^{*} \chi_{i} (s) = \frac{1}{|G|}|G|\chi_{i}(1)=\chi_{i}(1)=n_{i} 
\]
\end{proof}
\begin{corolario}
Los grados $n_{i}$ verifican la relaci'on $\sum_{i=1}^{h}n_{i}^{2}=|G|$.
\end{corolario}
\begin{proof}
En efecto, el corolario 3.3.1 asegura que $\varphi = \sum n_{i} \chi_{i}$ y aplicando ambos miembros a dicho corolario resulta $|G|=\sum_{i=1}^{h} n_{i}^{2}$. 
\end{proof}\\
\\
Este resultado se puede usar cuando se buscan las representaciones irreducibles de $G$; supongamos construidas representaciones irreducibles no isomorfas dos a dos, de grados $n_{1}, \ldots ,n_{k}$; a fin de que sean todas las representaciones irreducibles de $G$ (salvo isomorfismos) es necesario y suficiente que $n_{1}^{2}+ \ldots + n_{k}^{2}=|G|$.
\section{N'umero de representaciones irreducibles.}
Una aplicaci'on $f$ definida en $G$ se llama \emph{central} si $f(tst^{-1})=f(s)$ cualesquiera que sean $s, t \in G$.
\begin{proposicion}
Sea $f$ una funci'on central definida en $G$, y $\rho: G \rightarrow GL(V)$ una representaci'on lineal de $G$. Sea $\rho_{f}$ el endomorfismo de $V$ definido por:
\[
\rho_{f}=\sum_{t \in G} f(t)\rho (t)
\]
Si $V$ es irreducible, de grado $n$ y de car'acter $\chi$, $\rho_{f}$ es una homotecia de raz'on $\lambda$, donde
\[
\lambda = \frac{1}{n}\sum_{t \in G} f(t)\chi_{t}=\frac{|G|}{n}(f^{*}|_{\chi})
\]
\end{proposicion}
\begin{proof}
Calculamos $\rho (s)^{-1}\rho_{f}\rho (s) $:
\[
\rho (s)^{-1}\rho_{f}\rho (s)=\sum_{t \in G}f(t)\rho(s)^{-1}\rho(t)\rho(s)=\sum_{t \in G}f(t)\rho(s^{-1}ts)
\]
y poniendo $u=s^{-1}ts$, resulta
\[
\rho (s)^{-1}\rho_{f}\rho (s)=\sum_{u \in G}f(sus^{-1})\rho (u)=\sum_{u \in G}f(u)\rho (u)=\rho_{f}
\]
De modo que $\rho_{f}\rho(s)=\rho(s)\rho_{f}$. Esta igualdad implica que $\rho_{f}$ es una homotecia de raz'on $\lambda$. La traza de $\lambda$ es $n \lambda$; y la de $\rho_{f}$ es
\[
\sum_{t \in G}f(t)Tr(\rho (t))=\sum_{t \in G}f(t)\chi (t)
\]
de donde
\[
\lambda = \frac{1}{n}\sum_{t \in G} f(t)\chi_{t}=\frac{|G|}{n}(f^{*}|_{\chi})
\]
\end{proof}\\
\\
Introduzcamos ahora el espacio vectorial $H$ de las funciones centrales de $G$. Los caracteres 
$\chi_{1}, \ldots ,\chi_{k}$ de las representaciones irreducibles de $G$ son elementos de $H$.
\begin{teorema}
Los caracteres $\chi_{1}, \ldots ,\chi_{h}$ forman una base ortonormal de $H$.
\end{teorema}
\begin{proof}
El teorema 3.3.1 demuestra que $\chi_{1}, \ldots ,\chi_{h}$ es un sistema ortonormal de $H$. Falta demostrar que este sistema es completo, es decir, que todo elemento de $H$ ortogonal a los $\chi_{i}$ es nulo. Para ello, sea $f$ un elemento de $H$ ortogonal a los $\chi_{i}$. Si $\rho : G \rightarrow GL(V)$ es una representaci'on de $G$, ponemos
\[
\rho_{f}^{*}=\sum_{t \in G}f(t)^{*}\rho (t)
\]
La proposici'on 3.4.1 muestra que $\rho_{f}^{*}$ es nula si $V$ es irreducible; descomponiendo en suma directa concluimos que $\rho_{f}^{*}$ es siempre nula. Aplicando este resultado a la representaci'on regular $R$ y calculando la imagen del vector $e_{1}$ de la base por $\rho_{f}^{*}$:
\[
\rho_{f}^{*}e_{1}=\sum_{t \in G} f(t)^{*}\rho(t)e_{1}=\sum_{t \in G}f(t)^{*}e_{t}
\]
Como$\rho_{f}^{*}$ es nula, esta igualdad implica que $f(t)^{*}=0$ para todo $t$, de donde $f=0$, lo cual termina la demostraci'on.
\end{proof}\\
\\
Recordemos que dos elementos $t$ y $t`$ de $G$ se dicen conjugados si existe $s \in G$ tal que $t'=sts^{-1}$; esta relaci'on es de equivalencia y divide a $G$ en clases.
\begin{teorema}
El n'umero de representaciones irreducibles de $G$ (salvo isomorfismos) es igual al n'umero de clases de $G$.
\end{teorema}
\begin{proof}
Sean $C_{1}, \ldots ,C_{k}$ las clases de $G$. Una funci'on $f$ definida en $G$ es central si y solo si es constante en cada una de las clases $C_{1}, \ldots ,C_{k}$ y por tanto una tal funci'on est'a determinada por $k$ valores $\lambda_{1}, \ldots ,\lambda_{k}$, que se pueden poner al azar. Resulta de ello que la dimensi'on del espacio $H$ de funciones centrales es igual a $k$. Por otra parte, esta dimensi'on es igual, seg'un el teorema 3.4.1, al n'umero de representaciones irreducibles de $G$ (salvo isomorfismos).
\end{proof}\\
\\
Otra consecuencia del teorema 3.4.1 es que siendo $s\in G$, el n'umero de elementos de la clase de $s$ y $f_{s}$ la funci'on igual a 1 sobre esta clase e igual a0 en el complementario. Como esta funci'on es central, por el teorema 3.4.1 tendremos:
\[
f_{s}=\sum_{i=1}^{i=h}x_{i}\chi_{i}
\]
donde
\[
x_{i}=\langle \chi_{i} , f_{s} \rangle = \frac{c_{s}}{|G|}\chi_{i}(s)^{*}
\]
As'i pues, para todo $t \in G$
\[
f_{s}(t)=\frac{c_{s}}{|G|}\sum_{i=1}^{i=h}\chi_{i}(s)^{*}\chi_{i}(t)
\]
Si explicitamos resultan las formulas siguientes:(para $t=s$):
\[
\sum_{i=1}^{i=h}\chi_{i}(s)^{*}\chi_{i}(s)=\frac{|G|}{c_{s}}
\]
(para $t$ no conjugado de $s$):
\[
\sum_{i=1}^{i=h}\chi_{i}(s)^{*}\chi_{i}(t)=0
\]
\begin{ejemplo}
Sea $G$ el grupo de permutaciones de 3 letras. Entonces $|G|=6$, y tiene tres clases: el elemento identidad, tres transposiciones y dos permutaciones c'iclicas. Sea $t$ una transposici'on, y $c$ una permutaci'on c'iclica. Tenemos $t^{2}=1$, $c^{3}=1$, $tc=c^{2}t$; de donde hay solo dos caracteres de grado 1: el car'acter identidad $\chi_{1}$ y e car'acter $\chi_{2}$ que nos varia el signo de la permutaci'on.\\
El 'ultimo teorema nos muestra que existe otro car'acter irreducible $\theta$; si $n$ es su grado debemos tener $1+1+n^{2}=6$, por lo que $n=2$. Los valores de $\theta$ pueden ser deducidos del hecho de que $\chi_{1}+\chi_{2}+2\theta$ es el car'acter de la representaci'on regular de $G$. Obtenemos la siguiente tabla de caracteres de $G$:
\begin{table}[H]
\begin{center}
\begin{tabular}{|c|c|c|c|}
\hline
& 1 & t & c \\
\hline \hline
$\chi_{1}$ & 1 & 1 & 1 \\ 
$\chi_{2}$ & 1 & -1 & 1 \\ 
$\theta$ & 2 & 0 & -1 \\ 
\hline
\end{tabular}
\caption{Tabla de caracteres de G.}
\label{tabla:caracteresG}
\end{center}
\end{table}
\end{ejemplo}

\end{document}
